\chapter{Introduction}
\pagenumbering{arabic}


Instruction pipelining is an interestingly simple concept... But to truly understanding this idea of parallelism can be challenging. It feels like reading music without ever hearing a song. The flow, the stalls, the hazards—they were words without a melody. And many tools that try to help with visualizing these notes are outdated in today's day and age.
This thesis was born out of a desire to transform that silence into a beautiful symphony, crafting an instrument that would allow learners to not just see, but experience the song of instruction pipelining.

In contrast to how a non pipelined processor works, a pipelined processor executes multiple instructions simultaneously. One at each stage of the pipeline. This allows for a higher throughput of instructions, but also introduces a new set of challenges.

The problem is that there is no modern, accessible, user-friendly tool that allows for the visualization of instruction pipelining in a way that is easy to understand. Most tools are either outdated, too complex to use, or lack the necessary features to provide a comprehensive learning experience. This creates a barrier to entry, which makes it difficult for students to learn about this interesting topic.

To bridge the gap. This thesis aims to analyze existing solutions, identify their limitations, and propose a comprehensive solution in the form of a web-based instruction pipeline simulator.
% The paper is organized as follows: Chapter 2 provides an overview of the existing solutions and their limitations. Chapter 3 describes the proposed solution and its features. Chapter 4 discusses the implementation details of the simulator. Chapter 5 presents the results of the evaluation of the simulator. Finally, Chapter 6 concludes the paper and provides directions for future work.
