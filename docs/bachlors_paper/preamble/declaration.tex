\thispagestyle{empty}
\section*{Declaration}

Slovenská technická univerzita v Bratislave
Ústav počítačového inžinierstva a aplikovanej informatiky		Fakulta informatiky a informačných technológií
2024/2025




ZADANIE BAKALÁRSKEJ PRÁCE


Autor práce:	Martin Dinja
Študijný program:	informatika
Študijný odbor:	informatika
Evidenčné číslo:	FIIT-16768-120774
ID študenta:	120774
Vedúci práce:	Ing. Ján Hudec, PhD.
Vedúci pracoviska:	Ing. Katarína Jelemenská, PhD.

Názov práce:	
Simulátor pre prúdové spracovanie inštrukcií

Jazyk, v ktorom sa práca vypracuje:	
slovenský jazyk

Špecifikácia zadania:	
Pre potreby výučby v predmete Princípy počítačového inžinierstva je účelné používať názorný simulátor pre prúdové spracovanie inštrukcií na zvolenej architektúre procesora. Z hľadiska bezproblémovej použiteľnosti na rozdielnych platformách navrhnuté riešenie musí byť spracované ako webová aplikácia. Inšpiráciou pre vývoj by mal byť aj používaný simulátor MIPSim. Analyzujte princípy prúdového spracovania inštrukcií a existujúce riešenia simulátorov takéhoto spôsobu práce procesora. Navrhnite a implementujte webovú aplikáciu s funkcionalitou vlastnej koncepcie simulátora prúdového spracovania inštrukcií, ktorý bude umožňovať editovať program pre procesor so zvoleným inštrukčným súborom, spustiť, krokovať a vykonať simuláciu prúdového spracovania inštrukcií na procesore, nastavovať a prehliadať obsah pamäti, registrov procesora, vizuálne sledovať stav simulácie jednotlivých fáz prúdového spracovania inštrukcií. Spracujte podrobný návod na použitie navrhnutého simulátora. Riešenie overte, otestujte pre praktické použitie, vyhodnoťte dosiahnuté výsledky a navrhnite možnosti rozšírenia a zdokonalenia simulátora.

Rozsah práce:	
40

Termín odovzdania práce:	
12. 05. 2025
\newpage
\thispagestyle{empty}
\mbox{}
\newpage