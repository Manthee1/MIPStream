\thispagestyle{empty}
\section*{Anotácia}

\begin{minipage}[t]{1\columnwidth}%
Slovenská technická univerzita v Bratislave

FAKULTA INFORMATIKY A INFORMAČNÝCH TECHNOLÓGIÍ

Študijný program: Informatika
\newline

Autor: \myName

Bakalárska práca: Simulator na prúdove spracovanie inštrukcií

Vedúci bakalárskej práce: \mySupervisor

Máj 2025
\end{minipage}

\bigskip{}

Cieľom tejto práce je vyriešiť problém zastaraných a ťažko použiteľných simulátorov pre prúdove spracovanie inštrukcií -hlavne archaického MIPSIMu používaného v predmete Základy Počítačového Inžinierstva na Slovenskej technickej univerzita v Bratislave
v ústave počítačového inžinierstva a aplikovanej informatiky- 
zavedením nového simulátora. Je navrhnutý tak, aby ho mohli používať študenti a učitelia na lepšie pochopenie koncepcie prúdoveho spracovania inštrukcií a jeho vnútorného fungovania. Táto práca analyzuje koncept prúdoveho spracovania inštrukcií, existujúce simulátory a ich nedostatky. Navrhuje nový simulátor, ktorý tieto nedostatky rieši, a opisuje jeho implementáciu. Nový simulátor je implementovaný pomocou moderných webových technológií a je navrhnutý tak, aby bol užívateľsky prívetivý a ľahko použiteľný. Simulátor je hodnotený študentmi a učiteľmi s cieľom určiť jeho použiteľnosť a účinnosť pri výučbe a učení sa pipelingu inštrukcií.




\newpage{}\thispagestyle{empty}

\newpage
\thispagestyle{empty}
\mbox{}
\newpage



\newpage{}\thispagestyle{empty}\medskip{}

\thispagestyle{empty}

\section*{Annotation}

\begin{minipage}[t]{1\columnwidth}%
Slovak University of Technology Bratislava 

FACULTY OF INFORMATICS AND INFORMATION TECHNOLOGIES

Degree Course: \myStudyProgram
\newline

Author: \myName

Bachelor’s Thesis: \myTitle

Supervisor: \mySupervisor

\myDate%
\end{minipage}

\bigskip{}

The aim of this thesis is to address the problem of outdated and difficult to use simulators for instruction pipelining, -mainly the archaic MIPSIM used by the Principles of Computer Engineering course at Faculty of Informatics and Information Technologies, Slovak University of Technology in Bratislava- by introducing a new simulator. It is designed to be used by students and teachers to better understand the concept of instruction pipelining and its inner workings. This thesis analyzes the instruction pipelining concept, the existing simulators, their shortcomings, and proposes a new simulator that addresses those shortcomings, and describes its implementation. The new simulator is implemented using modern web technologies and is designed to be user friendly and easy to use. The simulator is evaluated by students and teachers to determine its usability and effectiveness in teaching and learning instruction pipelining.

\newpage{}


\thispagestyle{empty}
\mbox{}
\newpage

